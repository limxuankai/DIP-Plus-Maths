\documentclass[17 pt]{extarticle}

\usepackage{amsmath, amssymb, amsthm}
\usepackage{mathtools}

\newlength{\myeqskip}  \setlength{\myeqskip}{15pt}
\AtBeginDocument{%
    \setlength\abovedisplayskip{\myeqskip}%
    \setlength\belowdisplayskip{\myeqskip}%
    \setlength\abovedisplayshortskip{\myeqskip-\baselineskip}%
    \setlength\belowdisplayshortskip{\myeqskip}}

    \setlength{\jot}{10pt}

\title{Week 1 Tutorial Attempt}
\author{XK}
\date{\today}

\begin{document}


\maketitle

\tableofcontents

\newpage
\section{Question 1}

\subsection*{If $f(x) = 2x^3-x$, find $f(-1),f(0),f(x^2),f(\sqrt x), f(\frac{1}{x})$}
\vspace{10mm}
\subsection*{Answer:}

Just substitute the numbers accordingly
\begin{equation*}
\begin{split}
    f(-1) &= 2(-1)^3 - (-1) \\
    & = 2(-1) + 1 \\
    &= -2 + 1 \\
    &= -1 \; \; \blacksquare
\end{split}
\end{equation*}

\begin{equation*}
    \begin{split}
        f(0) &= 2(0)^3 - (0) \\
        & = 2(0) \\
        &= 0 \; \; \blacksquare
    \end{split}
    \end{equation*}

\begin{equation*}
        \begin{split}
            f(x^2) &= 2(x^2)^3 - (x^2) \\
            & = 2(x^6) - x^2 \\
            &= 2x^6 - x^2 \\
            &= x^2(2x^4 - 1) \; \; \blacksquare
        \end{split}
\end{equation*}

\begin{equation*}
    \begin{split}
        f(\sqrt x) &= 2(\sqrt x)^3 - (\sqrt x) \\
        & = 2(x^ \frac{3}{2}) - \sqrt x \\
        &= 2x^ \frac{3}{2} - x^ \frac{1}{2} \\
        &= x^ \frac{1}{2} (2x - 1) \; \; \blacksquare
    \end{split}
\end{equation*}

\begin{equation*}
    \begin{split}
        f \biggl \lparen \frac{1}{x} \biggr \rparen &= 2\biggl \lparen \frac{1}{x^3} \biggr \rparen - \biggl \lparen \frac{1}{x} \biggr \rparen \\
        & = 2 \biggl \lparen \frac{1}{x^3} \biggr \rparen - \frac{1}{x} \\
        &= \frac{2}{x^3} - \frac{1}{x} \\
        &= \frac{1}{x} \biggl \lparen \frac{2}{x^2} - 1 \biggr \rparen \; \; \blacksquare
    \end{split}
\end{equation*}

\newpage
\section{Question 2}

\subsection*{If $f(x) = \begin{cases} x^2+1, & \text{if } x \leq 0 \\
\sqrt x, & \text{if } x > 0 \end{cases}$, find $f(-2), f(0)$ and $f(1)$}

\vspace{10mm}

\subsection*{Answer:}

Same as Q1, proper substitution needs to be performed here. 
\begin{equation*}
    \begin{split}
        f(-2) &= 2(-2)^2 + 1 \; \; \; \textit{$x <  0$ hence take 1st option} \\
        & = 2(4) + 1 \\
        &= 7 \; \; \blacksquare
    \end{split}
    \end{equation*}

    \begin{equation*}
        \begin{split}
            f(0) &= 2(0)^2 + 1 \; \; \; \textit{$x = 0$ hence take 1st option} \\
            &= 1 \; \; \blacksquare
        \end{split}
        \end{equation*}

        \begin{equation*}
            \begin{split}
                f(1) &= 2(1)^2 + 1 \; \; \; \textit{$x > 0$ hence take 2nd option} \\
                &= 2(1) + 1 \\
                &= 3 \; \; \blacksquare
            \end{split}
            \end{equation*}
\section{Question 3}

\subsection*{Find domain of the following function: \newline \\ $f(x) =\sqrt{x-2} + \sqrt{4-x}$
\newline \\ $f(x) = \frac{\sqrt{x+2} + \sqrt{2-x}}{x^3-x}$}

\vspace{10mm}
\subsection*{Answer}

Honestly, I have a problem with this question. There is no indication of what the field is. Is it $\mathbb{R}$ or $\mathbb{C}$ field? Or is it only for $\mathbb{N}$? But most likely it will be either real or complex field. As 
such I'll write both fields for this question. Going forward, the domain is just what numbers can fit into this function to produce an output within the field.

\begin{equation*}
    \begin{split}
        f(x) =\sqrt{x-2} + \sqrt{4-x} \\
        \mathbb{C} \; \text{field for domain:} (- \infty, \infty) \\
        \mathbb{R} \; \text{field for domain:} [2,4] \\
         \blacksquare
    \end{split}
    \end{equation*}

So in the complex field, we don't care. Honestly, the square roots means nothing. You might as well eat them. But once we reached the 
real field, $\sqrt{x}$ matters more to us. Looking at the first part of $\sqrt{x-2}, \; x-2 \geq 0$ for the real answers hence $x \geq 2$. Now onto the second part, $\sqrt{4-x}, \;
4-x \leq 0$ hence $x \leq 4$. This only leaves us with the domain between 2 and 4 or $ [2,4] $

\vspace{5mm}

\begin{equation*}
    \begin{split}
        f(x) &= \frac{\sqrt{x+2} + \sqrt{2-x}}{x^3-x} \\
        &= \frac{\sqrt{x+2} + \sqrt{2-x}}{x(x^2 - 1)} \\
        &= \frac{\sqrt{x+2} + \sqrt{2-x}}{x(x+1)(x-1)}
    \end{split}
    \end{equation*}

\vspace{5mm}

\begin{equation*}
    \begin{split}
        \mathbb{C} \; \text{field for domain:} \; \mathbb{R} \neq -1,0,1 \\
        \mathbb{R} \; \text{field for domain:} \; [-2,2] \neq -1, 0, \\ \blacksquare 
    \end{split}
    \end{equation*}

This function is a little bit more ugly. Fractions with variables as the denominator are always going to have some problems. So even in the complex field, 
we need to take note of the denominator. I factorised $ x^3-x $ to $ x(x+1)(x-1)$ such that any of these values cannot hit 0. As such, we can have any values except from -1 to 1.
Since we cannot have -1 or 1, we need to use open intervals (). And I just remembered, the values between -1 to 1 are valid except for 0, which means more more unions. But that's stupid.
Instead, I'll write out the three invalid numbers within here. 

\vspace{5mm}

For the real field, it is easier. Just need the values inside the square roots to not be negative. So $ x+2 \geq 0 => x \geq -2 \newline 2-x \leq 0  => 2 \leq x $. This gives us 
the closed range of $ [-2 \leq x \leq 2]$. So all we have to do is exclude the same values as before and we're done. So tiring, isn't it?

\section{Question 4}

\subsection*{4a) Find $f(7)$}
\vspace{5mm}
\subsection*{Answer}

Man, just read the graph. $f(7)=3 \\ \blacksquare$
\vspace{10mm}
\subsection*{4b) Find the values of $x$ corresponding to the point(s) on the
graph of f located at a height of $5$ units above the $x$-axis}
\vspace{5mm}
\subsection*{Answer}

Once again, read the graph. This time read the graph when $f(x)=5$. $x = 4,6 \\ \blacksquare$
\vspace{5mm}

\subsection*{4c) Find the point on the x-axis at which the graph of f crosses it. What is $f(x)$ at this point?}
\vspace{5mm}
\subsection*{Answer}

I love reading graph for a third time!!! This time we need to read when $y=0$. $f(x)$ at that point is $(2,4) \\ \blacksquare$

\vspace{10mm}

\subsection*{4d) Find the domain and range of $f(x)$?}
\vspace{5mm}
\subsection*{Answer}

Finally, here's a question that you have to think. Sadly, you don't think much. You take the max and min of x to be domain. Take the max and min
of y to be range. 

\begin{equation*}
    \begin{split}
        \text{Domain:} \; [-1,9] \\
        \text{Range:} \; [-2,6] \\ \blacksquare 
    \end{split}
    \end{equation*}

\section{Question 5}

\subsection*{Find domain and sketch graph of the function. What is its range? $f(x) = \begin{cases} -x+1, & \text{if } x \leq 1 \\
    x^2-1 & \text{if } x > 0 \end{cases}$}

\vspace{10mm}
\subsection*{Answer:}

Because I'm allegric to graphs, I'm not going to sketch it out. At least not yet until when I learn how to insert a graph into
Latex files. But let's consider the domain. I think it's kinda obvious, you can shove any number as \(x\) and the function will still 
\textit{function}. Get itit, it's a pun. Now laugh. 

\begin{equation*}
    \begin{split}
        \text{Thus the domain will be:} \; (- \infty, \infty) \\
    \end{split}
    \end{equation*}

What about range? Range will also be much like the domain no? Shove in the biggest number and get infinity for the \(x>0\). 
But the other one doesn't work. Any number \(x \leq 1\) will cause some problems. Since its \(-x\), when we shove negative infinity into \(x\), 
we get infinity once again. Hence, after some logical reasoning, the smallest number for \(-x+1\) must be achieved when \(x=1\) to get \(0\).

\begin{equation*}
    \begin{split}
        \text{Therefore, range will be} \; [0, \infty) \\ \blacksquare 
    \end{split}
    \end{equation*}


\enddocument{}