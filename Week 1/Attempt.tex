\documentclass[17 pt]{extarticle}

\usepackage{amsmath, amssymb, amsthm}
\usepackage{mathtools}

\newlength{\myeqskip}  \setlength{\myeqskip}{15pt}
\AtBeginDocument{%
    \setlength\abovedisplayskip{\myeqskip}%
    \setlength\belowdisplayskip{\myeqskip}%
    \setlength\abovedisplayshortskip{\myeqskip-\baselineskip}%
    \setlength\belowdisplayshortskip{\myeqskip}}

    \setlength{\jot}{10pt}

\title{Week 1 Tutorial Attempt}
\author{XK}
\date{\today}

\begin{document}


\maketitle

\tableofcontents

\newpage
\section{Question 1}

\subsection*{If $f(x) = 2x^3-x$, find $f(-1),f(0),f(x^2),f(\sqrt x), f(\frac{1}{x})$}
\vspace{10mm}
\subsection*{Answer:}

Just substitute the numbers accordingly
\begin{equation*}
\begin{split}
    f(-1) &= 2(-1)^3 - (-1) \\
    & = 2(-1) + 1 \\
    &= -2 + 1 \\
    &= -1 \; \; \blacksquare
\end{split}
\end{equation*}

\begin{equation*}
    \begin{split}
        f(0) &= 2(0)^3 - (0) \\
        & = 2(0) \\
        &= 0 \; \; \blacksquare
    \end{split}
    \end{equation*}

\begin{equation*}
        \begin{split}
            f(x^2) &= 2(x^2)^3 - (x^2) \\
            & = 2(x^6) - x^2 \\
            &= 2x^6 - x^2 \\
            &= x^2(2x^4 - 1) \; \; \blacksquare
        \end{split}
\end{equation*}

\begin{equation*}
    \begin{split}
        f(\sqrt x) &= 2(\sqrt x)^3 - (\sqrt x) \\
        & = 2(x^ \frac{3}{2}) - \sqrt x \\
        &= 2x^ \frac{3}{2} - x^ \frac{1}{2} \\
        &= x^ \frac{1}{2} (2x - 1) \; \; \blacksquare
    \end{split}
\end{equation*}

\begin{equation*}
    \begin{split}
        f \biggl \lparen \frac{1}{x} \biggr \rparen &= 2\biggl \lparen \frac{1}{x^3} \biggr \rparen - \biggl \lparen \frac{1}{x} \biggr \rparen \\
        & = 2 \biggl \lparen \frac{1}{x^3} \biggr \rparen - \frac{1}{x} \\
        &= \frac{2}{x^3} - \frac{1}{x} \\
        &= \frac{1}{x} \biggl \lparen \frac{2}{x^2} - 1 \biggr \rparen \; \; \blacksquare
    \end{split}
\end{equation*}

\newpage
\section{Question 2}

\subsection*{If $f(x) = \begin{cases} x^2+1, & \text{if } x \leq 0 \\
\sqrt x, & \text{if } x > 0 \end{cases}$, find $f(-2), f(0)$ and $f(1)$}

\vspace{10mm}

\subsection*{Answer:}

Same as Q1, proper substitution needs to be performed here. 
\begin{equation*}
    \begin{split}
        f(-2) &= 2(-2)^2 + 1 \; \; \; \textit{$x <  0$ hence take 1st option} \\
        & = 2(4) + 1 \\
        &= 7 \; \; \blacksquare
    \end{split}
    \end{equation*}

    \begin{equation*}
        \begin{split}
            f(0) &= 2(0)^2 + 1 \; \; \; \textit{$x = 0$ hence take 1st option} \\
            &= 1 \; \; \blacksquare
        \end{split}
        \end{equation*}

        \begin{equation*}
            \begin{split}
                f(1) &= 2(1)^2 + 1 \; \; \; \textit{$x > 0$ hence take 2nd option} \\
                &= 2(1) + 1 \\
                &= 3 \; \; \blacksquare
            \end{split}
            \end{equation*}
\end{document}